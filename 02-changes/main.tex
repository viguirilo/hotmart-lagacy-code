\section{Start-Up ou não, Eis a Questão}

Muitas empresas intitulam-se como ``Start-Up'' sem saber de fato o que isso significa. Este termo pode estar relacionado à quantidade de pessoas que trabalham nela, ao baixo orçamento ou a ambiente de trabalho flexível, dentre outros.

Uma Start-Up também pode se caracterizar por seu \emph{modus operandi}, ou seja, um negócio possivelmente rentável mas que não atingiu maturidade suficiente para seguir com as próprias pernas e utiliza-se da estratégia \textbf{``erre rápido, conserte mais rápido ainda''}. 

Assim que uma Start-Up atinge seu nível de maturidade, ela começa a investir em pessoas e no ambiente de trabalho por saber que isso irá trazer bons resultados mas por que a Start-Up mas muitas vezes negligencia um dos ses maiores patrimônios, o software.

Talvez duas teorias possam ser plausíveis em casos como esses:
\begin{enumerate}
    \item O software é invisível: a empresa deve estar sempre vigilante para entender quando o software requer uma refatoração ou reconstrução.
    \item Time em que está ganhando não se mexe: essa é uma falácia pois sempre podemos melhorar um processo quando (i) revisamos o processo e o conseguimos melhorar ou quando (ii) não conseguimos melhorar o processo mas melhoramos nossa percepcão sobre ele. Ou seja, sempre há uma melhora.

\end{enumerate}