\section{O Poder do Conhecimento}

Confesso que toda essa situação me deixava inquieto e após muita reflexão, tentei buscar na internet alguma resposta. Dentre o jargão corporativo norte-americano temos o padrão $C(x)O \; | \; x \in [A...Z]$ como por exemplo \emph{CEO}\footnote{Chief of Executive Officer}, \emph{CTO}\footnote{Chief of Technology Officer}, \emph{CFO}\footnote{Chief of Financial Officer} e tantos outros. 

Mas o que me chamou mais a atenção foi o \textbf{K}, da \textbf{Chief Of Knowledge Officer}. Dentre suas atribuições, temos\footnote{https://searchcio.techtarget.com/definition/CKO}:

\begin{quotation}
    Diretor do conhecimento (CKO) é um título corporativo para a pessoa responsável por supervisionar a gestão do conhecimento dentro de uma organização. A posição do CKO está relacionada, mas é mais ampla do que a posição do CIO. A função do CKO é garantir que a empresa lucre com o uso eficaz dos recursos de conhecimento. Os investimentos em conhecimento podem incluir funcionários, processos e propriedade intelectual; um CKO pode ajudar uma organização a maximizar o retorno sobre o investimento (ROI) sobre esses investimentos.
    Além disso, um CKO pode ajudar uma organização a:
    \begin{itemize}
        \item Maximizar o retorno sobre o investimento (ROI) em conhecimento.
        \item Maximizar os benefícios de ativos intangíveis, como marca e relacionamento com o cliente.
        \item Repetir os sucessos, analise e aprendizagem com os fracassos.
        \item Promover as melhores práticas.
        \item Promover a inovação.
        \item Evitar a perda de conhecimento que pode resultar da perda de pessoal.
    \end{itemize}
\end{quotation}

Portanto, o que temos hoje como um mantra talvez pudesse formar o quarto pilar: O Pilar do Conhecimento. É muito mais fácil sustentar qualquer coisa com 4 pilares do que com apenas 3.

Uma característica bem interessante do Knowledge Officer é
\begin{quotation}
    \textbf{EXTRAIR O QUE AS PESSOAS TÊM DE MELHOR}
\end{quotation}
Ou seja, esse Novo Pilar além de estar sempre de braços abertos a toda e qualquer iniciativa, dando acolhimento e direcionamento adequado, pode contribuir para reescrever a missão da empresa, veja como:
\begin{quotation}
    \textbf{EXTRAIR O QUE AS PESSOAS TÊM DE MELHOR PARA PERMITIR QUE AS PESSOAS POSSAM VIVER DE SUAS PAIXÕES}
\end{quotation}

Isso seria excelência de ponta a ponta.

Para finalizar, para que tudo isso possa ser possível, seria necessária a criação de uma nova Torre, chamada de Torre do Conhecimento onde agregaria todos os Especialistas, Product Managers, Designers e Writers.
Essa Torre deveria receber as demandas da empresa, levantar requisitos, modelar entidades, elaboras layouts gráficos e textuais, e deve entregar um projeto no Jira já devidamente mensurado e organizado pronto para o time de desenvolvimento.
