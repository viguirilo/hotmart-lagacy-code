\section{O Poder do Conhecimento}

Confesso que toda essa situação me deixava inquieto e após muita reflexão, tentei buscar na internet alguma resposta. Dentre o jargão corporativos norte-americano temos o padrão $C(x)O \; | \; x \in [A...Z]$ como por exemplo \emph{CEO}\footnote{Chief of Executive Officer}, \emph{CTO}\footnote{Chief of Technology Officer}, \emph{CFO}\footnote{Chief of Financial Officer} e tantos outros. 

Mas o que me chamou mais a atenção foi o \textbf{K}, da \textbf{CHIEF OF KNOWLEDGE OFFICER}. Dentre suas atribuições, temos\footnote{https://searchcio.techtarget.com/definition/CKO}:

\begin{quote}
    Chief knowledge officer (CKO) is a corporate title for the person responsible for overseeing knowledge management within an organization. The CKO position is related to, but broader than, the CIO position. The CKO's job is to ensure that the company profits from the effective use of knowledge resources. Investments in knowledge may include employees, processes and intellectual property; a CKO can help an organization maximize the return on investment (ROI) on those investments.
    Furthermore, a CKO can help an organization to:

    \begin{itemize}
        \item Maximize the return on investment (ROI) in knowledge.
        \item Maximize benefits from intangible assets, such as branding and customer relationships.
        \item Repeat successes and analyze and learn from failures.
        \item Promote best practices.
        \item Foster innovation.
        \item Avoid the loss of knowledge that can result from loss of personnel.
    \end{itemize}
\end{quote}

Portanto, o que temos hoje como um mantra talvez pudesse formar o quarto pilar: O Pilar do Conhecimento. E afinal de contas, é muito mais sustentar qualquer coisa com 4 pilares do que com apenas 3.

Duas outra característica bem interessante do Knowledge Officer são: 
\begin{itemize}
    \item Manter todo o conhecimento centralizado e organizado
    \item Extrair o que as pessoas têm de melhor a oferecer
\end{itemize}
Ou seja, esse Officer deve estar sempre de braços abertos a toda e qualquer iniciativa, dando acolhimento e direcionamento adequado. e o melhor de tudo, poderemos reescrever a missão da empresa, veja como:
\begin{quote}
    \textbf{EXTRAIR O QUE AS PESSOAS TÊM DE MELHOR A OFERECER PARA
    PERMITIR QUE AS PESSOAS POSSAM VIVER DE SUAS PAIXÕES}
\end{quote}

Isso seria excelência de ponta a ponta.

Para finalizar, para que tudo isso possa ser possível, seria necessária a criação de uma nova Torre, chamada de Torre do Conhecimento onde agregaria todos os Especialistas, Product Managers, Designers e Writers.
Essa Torre deveria receber as demandas da empresa, levantar requisitos, modelar entidades, elaboras layouts gráficos e textuais, e deve entregar um projeto no Jira já devidamente mensurado e organizado pronto para o time de desenvolvimento.
