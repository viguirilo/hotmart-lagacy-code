\section{O Poder do Conhecimento}

Confesso que toda essa situação me deixava inquieto e após muita reflexão, busquei na internet alguma resposta e foi dentro do jargão corporativo norte-americano\footnote{$C(x)O \; | \; x \in [A...Z]$} onde achei a resposta: \textbf{CKO - Chief Of Knowledge Officer}. Dentre suas atribuições, temos\footnote{https://searchcio.techtarget.com/definition/CKO}:
\begin{quotation}
    Diretor do conhecimento (CKO) é um título corporativo para a pessoa responsável por supervisionar a gestão do conhecimento dentro de uma organização. A posição do CKO está relacionada, mas é mais ampla do que a posição do CIO. A função do CKO é garantir que a empresa lucre com o uso eficaz dos recursos de conhecimento. Os investimentos em conhecimento podem incluir funcionários, processos e propriedade intelectual; um CKO pode ajudar uma organização a maximizar o retorno sobre o investimento (ROI) sobre esses investimentos.
    Além disso, um CKO pode ajudar uma organização a:
    \begin{itemize}
        \item Maximizar o retorno sobre o investimento (ROI) em conhecimento.
        \item Maximizar os benefícios de ativos intangíveis, como marca e relacionamento com o cliente.
        \item Repetir os sucessos, analise e aprendizagem com os fracassos.
        \item Promover as melhores práticas.
        \item Promover a inovação.
        \item Evitar a perda de conhecimento que pode resultar da perda de pessoal.
    \end{itemize}
\end{quotation}

Portanto, o que temos hoje como um mantra talvez pudesse formar o quarto pilar: \textbf{O Pilar do Conhecimento}, o que de uma forma lúdica seria mais fácil sustentar a empresa.

Uma característica bem interessante do Knowledge Officer é
\begin{quotation}
    \textbf{Extrair o que as pessoas têm de melhor}
\end{quotation}
Esse Novo Pilar além de estar sempre de braços abertos a toda e qualquer iniciativa, dando acolhimento e direcionamento adequado, poderia contribuir para reescrever a missão da empresa, veja como:
\begin{quotation}
    \textbf{Extrair o que as pessoas têm de melhor para permitir que as pessoas possam viver de suas paixões}\footnote{excelência de ponta a ponta}
\end{quotation}


Finalmente, para viabilizar todas essas ideias, seria interessante haver uma nova Torre, chamada Torre do Conhecimento, que deveria receber as demandas da empresa, levantar requisitos, modelar entidades, elaboras layouts gráficos e textuais\footnote{Localizados em todos os idiomas}, e deve entregar um projeto no Jira mensurado e organizado pronto para ser escalonado e desenvolvido.
Esta torre seria composta por:
\begin{itemize}
    \item Especialistas: direciona quando houver alguma dúvida ou conflito;
    \item Product Managers: registram todas as regras de negócio, criando Epic, História ou Tarefa
    \item Designers: registram todos layouts, criando Epic, História ou Tarefa
    \item Writers: registram layouts textuais, criado Epic, História ou Tarefa
    \item Time de Localização e Tradução: para cada layout textual, deve criar uma tarefa linkada contendo as traduções e localizações
    \item Gerentes e Coordenadores: antes do projeto iniciar, devem mensurar o esforço de cada tarefa. Com o projeto em andamento, devem acompanhar a execução dos trabalhos a fim de melhorar as estimativas e deixá-las mais precisas em projetos futuros. 
    \item Quality Assurance: equipe multidisciplinar responsável por avaliar os resultados, desde testes quanto aplicação correta dos playbooks. Este time não deve ser único e depende da demanda de projetos. É interessante que haja uma rotação para que todos os desenvolvedores adquiram uma bagagem de QA.
\end{itemize}