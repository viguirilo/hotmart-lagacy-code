\chapter{Introdução}

De antemão, gostaria de dizer que aou um Trooper apaixonado e advogo sempre em causa da Hotmart em todos os lugares que vou, quer seja falando bem da empresa, quer seja incentivando meus amigos a participarem como produtores ou como troopers. Por isso, tomo a liberdade de tecer algumas críticas construtivas sempre prezando pelo crescimento da empresa.

Como é de conhecimento geral, a Hotmart é um excelente lugar para se conviver, contar histórias, dividir momentos, criar novas soluções. Porém, não podemos dizer o mesmo sobre o nosso ambiente de trabalho. Uai, como assim? Bem, euu entendo que ambiente de trabalho não é somente aquele onde vc convive, mas aquele onde aquele você executa seu trabalho da forma como deveria ser.

Eu, como desenvolvedor, não posso afirmar que me sinto completamente seguro ou à vontade em meu local de trabalho. Afinal de contas, há regras de negócio estão espalhadas por diversos trechos de código ou na mente da quem desenvolveu, tenho que revisar códigos que nem sei muito bem do que se trata, sou obrigado a colocar qualidade em algo onde não há referência de qualidade, recebo mais 100 emails de alertas diariamente em minha caixa de email, quando estou de plantão, tenho que lidar com projetos os quais conheço pouco ou quase nada, e assim por diante, tenho que parar meu trabalho frequentemente para atender tickets iu reuniões que não são do meu interesse. 

Mas agora é que vem a melhor parte:
\begin{enumerate}
    \item Tudo que foi dito anteriormente é bastante comum
    \item Muitos outros experimentam essa insatisfação mas não acreditam na mudança
    \item As lideranças entendam que mudar o core business é preciso ou dão pouca importância para osso
\end{enumerate}

\section{A Maturidade}

Toda empresa ou indivíduo quando que nasce, passa pelo processo do seu descobrimento profissional ou individual onde toma escolhas acreditando estar indo na direção correta. Muitas dessas escolhas, em um primeiro momento, podem parecer boas decisões. Outras porém, podem se revelar escolhas ruins. O fato é que não é possível perceber quais são as boas e as más decisões. Algumas delas, por piores que sejam, pelo menos ajudam a chegar em algum caminho. Assim que atingimos esse caminho, conseguimos acertar mais do que errar, ou a amplitude de nossas ações diminui drasticamente e as coisas vão se moldando de uma uma forma mais sólida do que antes. Este fenômeno é conhecido como maturidade.

\begin{figure}[H]
    \centering
    \includegraphics[scale=0.37,keepaspectratio=true]{images/01.png}
    \caption{Processo de amadurecimento}
    \label{mature_process}
\end{figure}

\section{O Monte Everest}

\section{Em time que está ganhando não se mexe, será?}
% Falar sobre que softeare mao eh visro
% colocar aqui o exemplo de se fazer uma