\documentclass[a4paper,titlepage,twoside,openright]{report}
\usepackage{graphicx} % para da cor
\usepackage[brazil]{babel} % Para traduzir nomes que aparecem em inglês na estrutura do documento.
\usepackage[utf8]{inputenc} % This package allows the user to specify an input encoding
\usepackage[T1]{fontenc} % Permite que o LaTeX compreenda a acentuação feita direto pelo teclado. 
\usepackage{amsfonts} % Define alguns estilos de letras para o ambiente matemático
\usepackage{fancyhdr} % Para fazer cabeçalhos personalizados
\usepackage{hyperref} % Para tornar os links clicáveis
\hypersetup{
    colorlinks=true,
    linkcolor=blue,
    filecolor=blue,
    citecolor=blue,
    urlcolor=blue,
}
\usepackage{microtype}
\DisableLigatures{encoding = *, family = * }
\title{Construindo APIs baseadas em Spring Boot}
\author{Vitor G. R. Lopes}
\begin{document}
    \maketitle
    \tableofcontents
    \include{NETWORKING/networking.tex}
    \chapter{Hypertext Transfer Protocol -- HTTP}

\section{Os Verbos HTTP}
\subsection{Criar -- POST}
\subsection{Ler -- GET}
\subsection{Atualizar -- PUT}
\subsection{Atualizar -- PATCH}
\subsection{Deletar -- DELETE}
\subsection{HEAD}
\subsection{OPTIONS}
    \chapter{Aplicação Cliente Servidor}

\section{Códigos de status de respostas HTTP}
\subsection{Respostas de informação 1.x.x}
\subsection{Respostas de sucesso 2.x.x}
\subsection{Redirecionamentos 3.x.x}
\subsection{Erros do cliente 4.x.x}
\subsection{Erros do servidor 5.x.x}
% https://developer.mozilla.org/pt-BR/docs/Web/HTTP/Status
    \chapter{Banco de Dados -- Um Lugar Sagrado}

\section{Relacional ou Não Relacional -- Eis a Questão}
    \chapter{API -- Application Programming Interface}
    \chapter{O Padrão SOAP}
    \chapter{O Padrão REST}
    \chapter{Hello World (Olá Mundo)}
\end{document}